%----------------------------------------------------------------------------------------
%	WORK EXPERIENCE SECTION
%----------------------------------------------------------------------------------------

\section{experience}

\begin{entrylist}
%------------------------------------------------
\entry
  {2016\\\faChevronDown\\{present}}
  {Amazon Prime Air}
  {Seattle, WA, USA}
  {\jobtitle{Hardware Development Manager - Aircraft Electronics}
\begin{itemize}[leftmargin=12pt]
  \item Recruited, trained, and managed a cross-functional team of electrical, mechanical, system, and support engineers. Completed over 60 interviews.
  \item Managed org-wide electrical engineering support resources, including component librarian services, Altium ECAD administration, design standards documentation, and quality \& reliability processes/workflows.
  \item Oversaw the requirements derivation, development, testing, and qualification of key Avionics subsystems. Liaison between the Design Engineering team and the Systems, Reliability, Hardware Test, and Manufacturing Engineering teams.
\end{itemize}
\jobtitle{Sr. Hardware Development Engineer}
\begin{itemize}[leftmargin=12pt]
  \item Owner of several vehicle subsystems such as vehicle power distribution, multiple sensor systems, and ground station equipment.
  \item Supported the integration of Avionics subsystems into the vehicle. Developed installation checklists and provided on-call support to flight ops.
  \item Liaison between engineering teams to ensure that hardware designs maximize the efficiency of firmware development. Examples include boundary scan, standardized debug interface and indicators, and consistent pinout.
  \item Managed an external vendor relationship to develop a custom product to reduce the size, weight, and power consumption of the power regulation subsystem.
\end{itemize}
\jobtitle{Hardware Development Engineer}
\begin{itemize}[leftmargin=12pt]
  \item Oversaw the entire hardware process from design through to manufacturing: Component selection, PCB design, prototypes, and testing.
  \item Developed manufacturing, assembly, and testing procedures to ensure that high quality products are delivered to the vehicle program.
  \item Produced a standardized set of design artifacts for use across vehicle subsystems, ensuring consistent quality and reliability for Avionics hardware.
  \item Firmware development for a RF radio link to the ground. Developed a BSP package and drivers for the common vehicle microcontroller. Planning and development of unit, regression, and integration tests.
\end{itemize}
}
%------------------------------------------------
\entry
  {2010\\\faChevronDown\\2016}
  {SPARQ Systems}
  {Kingston, Ontario, Canada}
  {\jobtitle{Lead Product Developer}
\begin{itemize}[leftmargin=12pt]
  \item Recruited and trained new employees to grow the group from just myself to a team of six highly talented developers and engineers.
  \item Actively involved in high-level market research, feature requirements derivation, and product requirements specifications.
  \item Developed an in-house embedded Linux device utilizing advanced Zigbee communication, USB, 802.11 WiFi, and a WebSocket API to connect to cloud servers.
  \item Built an AWS-based monitoring and control solution using web technologies.
  \item Oversaw the entire hardware process from design through to manufacturing, including component selection, PCB design, mechanical design, prototyping, and testing. Developed manufacturing, assembly, and testing procedures to ensure that high quality products are delivered to customers.
  \item Led and supported the deployment of field trials at sites across North America.
  \item Coordinated multiple teams and external contractors working on key projects.
  \item Developed a novel Power Line Communication protocol using Forward Error Correcting codes for robust communication with solar microinverters.
\end{itemize}
}
%------------------------------------------------
\end{entrylist}